\documentclass[12pt]{article}

\include{preamble}

\newtoggle{professormode}
\toggletrue{professormode} %STUDENTS: DELETE or COMMENT this line



\title{MATH 341/641 Fall \the\year{} Homework \#3}

\author{Professor Adam Kapelner} %STUDENTS: write your name here

\iftoggle{professormode}{
\date{Due by email 11:59PM October 8, \the\year{} \\ \vspace{0.5cm} \small (this document last updated \today ~at \currenttime)}
}

\renewcommand{\abstractname}{Instructions and Philosophy}

\begin{document}
\maketitle

\iftoggle{professormode}{
\begin{abstract}
The path to success in this class is to do many problems. Unlike other courses, exclusively doing reading(s) will not help. Coming to lecture is akin to watching workout videos; thinking about and solving problems on your own is the actual ``working out.''  Feel free to \qu{work out} with others; \textbf{I want you to work on this in groups.}

Reading is still \textit{required}. For this homework set, review MATH 340 concepts: random variables, PMF's, PDF's and the normal distribution. 

The problems below are color coded: \ingreen{green} problems are considered \textit{easy} and marked \qu{[easy]}; \inorange{yellow} problems are considered \textit{intermediate} and marked \qu{[harder]}, \inred{red} problems are considered \textit{difficult} and marked \qu{[difficult]} and \inpurple{purple} problems are extra credit. The \textit{easy} problems are intended to be ``giveaways'' if you went to class. Do as much as you can of the others; I expect you to at least attempt the \textit{difficult} problems. \qu{[MA]} are for those registered for 621 and extra credit otherwise.

This homework is worth 100 points but the point distribution will not be determined until after the due date. See syllabus for the policy on late homework.

Up to 5 points are given as a bonus if the homework is typed using \LaTeX. Links to instaling \LaTeX~and program for compiling \LaTeX~is found on the syllabus. You are encouraged to use \url{overleaf.com}. If you are handing in homework this way, read the comments in the code; there are two lines to comment out and you should replace my name with yours and write your section. The easiest way to use overleaf is to copy the raw text from hwxx.tex and preamble.tex into two new overleaf tex files with the same name. If you are asked to make drawings, you can take a picture of your handwritten drawing and insert them as figures or leave space using the \qu{$\backslash$vspace} command and draw them in after printing or attach them stapled.

The document is available with spaces for you to write your answers. If not using \LaTeX, print this document and write in your answers. I do not accept homeworks which are \textit{not} on this printout. Keep this first page printed for your records.

\end{abstract}

\thispagestyle{empty}
\vspace{1cm}
NAME: \line(1,0){380}
\clearpage
}


\problem{Recall the iPhone survey from class where 10 of the $n=21$ surveyed responded that they have an iPhone.}

\begin{enumerate}

\easysubproblem{Create a 95\% CI for the true $\theta$, the proportion of iPhone ownership in 300-level STEM classes at QC. Use the CLT which implies $\Xbar$ is approximately normally distributed.}\spc{4}

\easysubproblem{Is this CI \emph{exact}? Yes / no and discuss.}\spc{2}

\intermediatesubproblem{Provide interpretations of this CI.}\spc{2}

\intermediatesubproblem{What is not a valid interpretation of this CI that you have wanted to be valid?}\spc{1}

\hardsubproblem{[MA] How would you construct an exact CI? Answer in English.}\spc{5}

\end{enumerate}


\problem{Remember the male student height data: $n=13$ and $\xbar = 68.85''$. We want to test at $\alpha = 1\%$ if the population that this sample was drawn from has a \emph{different} mean than the American male height mean of 69.92'' for ages 20-29 (this measurement is from more exact studides gathered from \href{https://www.healthline.com/health/average-height-for-men\#united-states}{this article}). You can assume that the true variance of the population if 4in$^2$ but \textit{do not assume} the DGP is normal!}

\begin{enumerate}

\easysubproblem{Write the alternative and null hypotheses.}\spc{1}

\intermediatesubproblem{Do we know the null distribution exactly? Why or why not?}\spc{2}

\easysubproblem{Write the approximate null sampling distribution on the original scale (i.e. in inches).}\spc{2}

\easysubproblem{Write the RET region as a set on the original scale.}\spc{2}

\easysubproblem{Is it possible to provide the exact $\prob{\text{Type I error}}$? Yes / no}\spc{-0.5}

\easysubproblem{What is the approximate $\prob{\text{Type I error}}$ in this test?}\spc{-0.5}

\easysubproblem{Is it possible to provide the exact p-val? Yes / no}\spc{-0.5}

\easysubproblem{Calculate the approximate p-val for this test.}\spc{2}

\easysubproblem{Is the dataset's estimate \qu{statistically significant}? Yes / no}\spc{-0.5}
\easysubproblem{Was this an exact test? Yes / no}\spc{-0.5}
\easysubproblem{Write a conclusion of this test in English.}\spc{2}

\intermediatesubproblem{Regardless of what the test's decision came out to be, assume $H_0$ is rejected. Is the dataset's estimate \qu{practically significant} (or \qu{clinically significant})? Discuss.}\spc{3}

\intermediatesubproblem{With a very large sample size, would $H_0$ always be rejected? Discuss.}\spc{3}

\easysubproblem{Imagine you could assume the sample was drawn $\iid$ from a normal distribution. Create a 95\% 2-sided CI for the mean height for all 300-level STEM courses at CUNY.}\spc{2}


\easysubproblem{Without assuming the sample was drawn $\iid$ from a normal distribution, create a 95\% 2-sided CI for the mean height for all 300-level STEM courses at CUNY. Hint: it will be the same!}\spc{2}

\hardsubproblem{What is the difference between these two CI's from the previous two problems?}\spc{3}


\easysubproblem{Does the CI include $\theta_0$? Yes / no / maybe}\spc{-0.5}
\intermediatesubproblem{Does the CI include $\theta$? Yes / no / maybe}\spc{-0.5}

\end{enumerate}




\problem{Here we will investigate MLE's and UMVUEs.}

\begin{enumerate}

\easysubproblem{Assume the DGP: $\Xoneton \iid \bernoulli{\theta}$. Find the MLE for $\theta$.}\spc{7}

\easysubproblem{Prove the MLE from from the previous problem is a UMVUE.}\spc{7}

\intermediatesubproblem{Assume the DGP: $\Xoneton \iid \exponential{\theta} := \theta e^{-\theta x} \indic{x>0}$. Find the MLE for $\theta$.}\spc{10}

\intermediatesubproblem{Now assume a different parameterization of the DGP, $\Xoneton \iid \exponential{1 / {\theta}}$. Find the MLE for $\theta$.}\spc{10}


\hardsubproblem{Prove the MLE from from the previous problem is a UMVUE.}\spc{5}

\end{enumerate}


\problem{Here we will get more practice with MM estimators}

\begin{enumerate}
\hardsubproblem{Consider the DGP $\Xoneton \iid \gammanot{\theta_1}{\theta_2}$. Below are some facts about this distribution that I took from \href{https://en.wikipedia.org/wiki/Beta_distribution}{wikipedia}:

\beqn
\gammanot{\theta_1}{\theta_2} &:=& \frac{\theta_2^{\theta_1}}{\Gammaf{\theta_1}}x^{\theta_1 - 1} e^{-\theta_2 x}\indic{x > 0}, ~~\support{X} = (0, \infty), ~~ \theta_1, \theta_2 \in (0, \infty), \\
\expe{X} &=&\int_0^{\infty} x f^{old}(x)dx = \frac{\theta_1}{\theta_2}, \\
\var{X} &=&\int_0^{\infty} (x - \expe{X})^2 f^{old}(x)dx = \frac{\theta_1}{\theta_2^2}
\eeqn

Find MM estimators for both parameters. Hint: leave expressions in terms of $\sigsqhat$.}\spc{8}

%set.seed(1984); paste0(round(rgamma(7,5,10),3), collapse = ", ")
\easysubproblem{Provide point estimates $\thetahathat_1$ and $\thetahathat_2$ for the unknown parameters $\theta_1$ and $\theta_2$ given the dataset 10.8, 8.5, 13.2, 9.1, 13.5, 11.2, 7.1 for the $\iid \gammanot{\theta_1}{\theta_2}$ DGP. No need to show work.}\spc{0}


\hardsubproblem{[MA] In Math 241 you learned about expectation and variance where expectation was a measure of central tendency of a distribution and variance is a measure of dispersion around that central tendency. The next most important metric for rv's is probably its \emph{skewness} defined as $\gamma := \skewness{X} := \expe{\tothepow{\frac{X - \expe{X}}{\sd{X}}}{3}}$ where SD refers to standard deviation. Skewness is technically the third standardized moment since $\frac{X - \expe{X}}{\sd{X}}$ is the distribution standardized and then the third power is taken. Skewness is a metric of which tail of the distribution is longer and by how much as seen in \href{https://codeburst.io/2-important-statistics-terms-you-need-to-know-in-data-science-skewness-and-kurtosis-388fef94eeaa}{this figure}. Since third powers are both positive and negative, skewness can be both positive and negative (and zero if the distribution is symmetric with right and left tails the same).  In class, we derived nonparametric MM estimators $\Xbar$ and $\sigsqhat$ for the expectation and variance (nonparametric meaning that the derivation for them was for \textit{all} iid DGP's). Show that the nonparametric MM estimator for skewness is:

\beqn
\hat{\gamma} = \sqrt{n}\frac{\sum_{i=1}^n (X_i - \Xbar)^3
}{
\tothepow{\sum_{i=1}^n (X_i - \Xbar)^2}{3/2}
}
\eeqn

Hint: assume a iid DGP with density / mass function $f(\theta_1, \theta_2, \theta_3)$ where $\theta_1$ is the expectation, $\theta_2$ is the variance and $\theta_3$ is the skewness.
}\spc{15}


\end{enumerate}

\problem{We will prove some of the main theorems of this class (and some other relevant facts) here.}


\begin{enumerate}


\easysubproblem{Prove the CRLB from scratch. Justify each step. List assumptions.}\spc{22}


\hardsubproblem{Prove that $S$, the score function over $n$ is asymptotically normal and find its mean. This fact will come in handy later in the course.}\spc{6}


\hardsubproblem{Prove that Fisher Information which is defined as $I(\theta) := \expe{\ell'(\theta; X)^2}$, the expected score squared, is equal to $\expe{-\ell''(\theta; X)}$. If you make any assumptions proving this, indicate it so. This is not easy. I suggest you try a bunch of manipulations that you saw performed in the proof of the CRLB and try out many of the definitions of $S$ from class.}\spc{16}


\end{enumerate}

\end{document}
