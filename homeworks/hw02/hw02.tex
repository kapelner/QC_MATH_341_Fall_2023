\documentclass[12pt]{article}

\include{preamble}

\newtoggle{professormode}
\toggletrue{professormode} %STUDENTS: DELETE or COMMENT this line



\title{MATH 341/641 Fall \the\year{} Homework \#2}

\author{Professor Adam Kapelner} %STUDENTS: write your name here

\iftoggle{professormode}{
\date{Due by email 11:59PM September 18, \the\year{} \\ \vspace{0.5cm} \small (this document last updated \today ~at \currenttime)}
}

\renewcommand{\abstractname}{Instructions and Philosophy}

\begin{document}
\maketitle

\iftoggle{professormode}{
\begin{abstract}
The path to success in this class is to do many problems. Unlike other courses, exclusively doing reading(s) will not help. Coming to lecture is akin to watching workout videos; thinking about and solving problems on your own is the actual ``working out.''  Feel free to \qu{work out} with others; \textbf{I want you to work on this in groups.}

Reading is still \textit{required}. For this homework set, review MATH 340 concepts: random variables, PMF's, PDF's, CDF's, binomial, and review MATH 241 concepts: the normal distribution. 

The problems below are color coded: \ingreen{green} problems are considered \textit{easy} and marked \qu{[easy]}; \inorange{yellow} problems are considered \textit{intermediate} and marked \qu{[harder]}, \inred{red} problems are considered \textit{difficult} and marked \qu{[difficult]} and \inpurple{purple} problems are extra credit. The \textit{easy} problems are intended to be ``giveaways'' if you went to class. Do as much as you can of the others; I expect you to at least attempt the \textit{difficult} problems. \qu{[MA]} are for those registered for 621 and extra credit otherwise.

This homework is worth 100 points but the point distribution will not be determined until after the due date. See syllabus for the policy on late homework.

Up to 5 points are given as a bonus if the homework is typed using \LaTeX. Links to instaling \LaTeX~and program for compiling \LaTeX~is found on the syllabus. You are encouraged to use \url{overleaf.com}. If you are handing in homework this way, read the comments in the code; there are two lines to comment out and you should replace my name with yours and write your section. The easiest way to use overleaf is to copy the raw text from hwxx.tex and preamble.tex into two new overleaf tex files with the same name. If you are asked to make drawings, you can take a picture of your handwritten drawing and insert them as figures or leave space using the \qu{$\backslash$vspace} command and draw them in after printing or attach them stapled.

The document is available with spaces for you to write your answers. If not using \LaTeX, print this document and write in your answers. I do not accept homeworks which are \textit{not} on this printout. Keep this first page printed for your records.

\end{abstract}

\thispagestyle{empty}
\vspace{1cm}
NAME: \line(1,0){380}
\clearpage
}

\problem{Here we will do a binomial exact test using the survey data from the 2021 class. We want to demonstrate that the iphone users in our class is \textit{greater} than the national average (which is 52.4\%). Recall that our data was as follows: for $n=20$, the $\thetahathat = 0.60$ where the estimator we chose was the sample proportion.}

\begin{enumerate}

\easysubproblem{Write down $H_a$ then $H_0$.}\spc{2}

\easysubproblem{Declare your $\alpha$ level desired for this test. You do not need to justify it. It is what you are comfortable with.}\spc{0}


\intermediatesubproblem{Because we want to show something is greater than a point value, it is called a right-tailed test. In any test, we need to find the distribution (or approximate the distribution of) the estimator under the null hypothesis. Because we will reject on the right, why is the most conservative value of $\theta$ to choose when deriving the null sampling distribution to be largest value in the null hypothesis region (in this case $\theta = \theta_0 = 0.524$)?}\spc{3}

\easysubproblem{Regardless of if you understood the previous question or not, what is the exact null sampling distribution (write the PMF)? Marked easy because you can copy from MATH 340 class notes.}\spc{1}

\easysubproblem{Draw the PMF of the null sampling distribution. Label all axes carefully and provide sufficient tick marks. Marked easy because you can copy from class. Probabilities are below rounded to the nearest 3 digits.}


\begin{table}[ht]
\centering\small
\begin{tabular}{rllllllllllllllllllllll}
0 & 1 & 2 & 3 & 4 & 5 & 6 & 7 & 8 & 9 & 10 & 11 & 12 & 13 \\
\hline
0 & 0 & 0 & 0.001 & 0.003 & 0.009 & 0.025 & 0.054 & 0.097 & 0.142 & 0.172 & 0.172 & 0.142 & 0.096 \\
14 & 15 & 16 & 17 & 18 & 19 & 20 & 21 \\ 
  \hline
0.053 & 0.023 & 0.008 & 0.002 & 0 & 0 & 0 & 0
\end{tabular}
\end{table}

~\spc{9}

\easysubproblem{Indicate the RET and the rejection region in the above illustration. Use your level $\alpha$. Everyone's answer may be different!}\spc{-0.5}


\easysubproblem{What is $\prob{\thetahat \notin RET}$ in this test with your level $\alpha$?}\spc{-0.5}

\intermediatesubproblem{Were you able to create a rejection region at your exact level of $\alpha$? Yes / no and why?}\spc{2}

\easysubproblem{What are all the possible sizes of this test (rounded to the number of digits in the PMF table).}\spc{2}
\easysubproblem{Run the test. Write your conclusion in English.}\spc{3}

\easysubproblem{What is $\prob{\text{Type I error}}$ in this test with your level $\alpha$?}\spc{-0.5}

\easysubproblem{What is the p-value of the estimate in this test?}\spc{1}

\easysubproblem{Assuming $\theta = 0.7$, what is the true sampling distribution? Write its PMF below.}\spc{3}

\intermediatesubproblem{Calculate the $\prob{\text{Type II error}}$ in this test.}\spc{2}

\easysubproblem{Easy only given the previous answer: calculate the power in this test.}\spc{1}

\end{enumerate}


\problem{Here we will review theory testing from a conceptual point of view. For each question, state whether the theory under consideration should become a null hypothesis or alternative hypothesis. If null, also write the alternative; if alternative also write the null.}

\begin{enumerate}

\easysubproblem{A new grand unified theory of physics.}\spc{1.5}

\easysubproblem{The latest conspiracy theory about the president.}\spc{1.5}

\easysubproblem{You are a shareholder in a pharmaceutical company. Your new drug cures cancer.}\spc{1.5}

\end{enumerate}


%\problem{Let $\Xoneton \iid \bernoulli{\theta}$, our DGP and we are focused on point estimation for $\theta$. We then choose the point estimator $\thetahat_{\text{bad}} = 0.1989$.}
%
%\begin{enumerate}
%
%\intermediatesubproblem{Graph the bias of $\thetahat_{\text{bad}}$ over all $\theta$. Label your axes.}\spc{4}
%
%\intermediatesubproblem{Graph the risk of $\thetahat_{\text{bad}}$ under squared error loss. Label your axes.}\spc{5}
%
%\hardsubproblem{Compare $\thetahat_{\text{bad}}$ to $\Xbar$ using the sup risk under squared error loss. How much better is $\Xbar$? With $n$ getting larger does it get even better?}\spc{4}
%
%\hardsubproblem{Assume that $\theta$ is drawn from $\Theta = [0, 1]$ uniformly. This breaks all of our rules about how $\theta$ is one fixed unknown value. But ignore that rule. This allows you to compare both MSE curves by computing $\int_0^1 MSE(\theta) d\theta$. How much better is $\Xbar$? With $n$ getting larger does it get even better?}\spc{7}
%
%\end{enumerate}


\end{document}
